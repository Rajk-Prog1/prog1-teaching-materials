\documentclass[12pt]{article}
% Dokumentum típusának megadása
% Ha kiveszed, akkor nem fogja compileolni a doksit.
% Lehetséges: 8pt, 9pt, 10pt, 11pt, 12pt, 14pt, 17pt, 20pt.

\usepackage[utf8]{inputenc}
% Input encoding megadása. Egyébként lehet más is pl. ASCII, Larin-2 stb.
% Nem feltétlenül kell, de a különböző compilerek miatt fontos lehet.

\usepackage[T1]{fontenc}
% Output font megadása
% Ha kiveszed, akkor nem kezeli jól az ékezetes szavakat és nem lesz másolható a PDF szövege

\usepackage[magyar]{babel}
% Magyar nyelvcsomag. Értelmezi hogy magyarul írsz.

\usepackage[outer=20mm, inner=20mm, top=20mm, bottom=20mm]{geometry}
% Megadja a margókat. Érdemes az elvárt szélességeket megadni, pl. szakdoga elvárás.

\renewcommand{\baselinestretch}{1.5}

\usepackage[usenames]{color}
% A szín elnevezések használatához kell. Lásd: https://www.overleaf.com/learn/latex/Using_colours_in_LaTeX
\definecolor{rajk}{RGB}{63,181,97}
% Csinálunk egy saját színt, amire később tudunk hivatkozni

\usepackage{hyperref}
\hypersetup{
    colorlinks=false,
    linkcolor=blue,
    filecolor=blue,      
    urlcolor=blue,
    pdftitle={Formális logika alapok}
    }

% \usepackage[nottoc]{tocbibind}
\usepackage{natbib}

% Not exists megjelenítése
\usepackage{amssymb}

\title{Formális logika alapok}
\date{}

\begin{document}

\maketitle

\section{Műveletek}

\begin{enumerate}
    \item $\neg$: tagadás (negáció)
    \item $\land$: és (konjukció)
    \item $\lor$: vagy (diszjunkció)
    \item $\oplus$: kizáró vagy (kizáró diszjunkció)
    \item $\Rightarrow$: implikáció
    \item $\Leftrightarrow$: ekvivalencia
\end{enumerate}

\section{Kvantorok}

\begin{enumerate}
    \item $\exists x$: létezik legalább egy $x$, amelyre igaz, hogy \dots
    \item $\forall x$: minden $x$-re igaz, hogy \dots
    \item $\nexists x$: nem létezik egy $x$ sem, amelyre igaz, hogy \dots
\end{enumerate}

\newpage
\section{Órai feladatok}

\noindent\textbf{Predikátumok:}

\begin{itemize}
    \item $F(x)$: $x$ férfi
    \item $N(x)$: $x$ nő
    \item $V(x, y)$: $x$ vonzónak tartja $y$-t
\end{itemize}

\noindent\textbf{Feladatok:}

\begin{enumerate}
    \item $d$-nek minden nő testzik:
    \begin{center}
        $\forall x(N(x) \Rightarrow V(d,x))$
    \end{center}
    
    \item $k$ egy biszexuális nő:
    \begin{center}
        $\exists x(N(x) \land V(k,x)) \land \exists y(F(y) \land V(k,y)) \land N(k)$
    \end{center}
    
    \item létezik aszexuális ember:
    \begin{center}
        $\exists x(\nexists y(V(x,y)))$\\
        vagy\\
        $\exists x(\forall y(\neg V(x,y)))$
    \end{center}
    
    \item mindenki biszexuális:
    \begin{center}
        $\forall x(\exists n(N(n) \land V(x,n)) \land \exists f(F(f) \land V(x,f)))$
    \end{center}
    
    \item csak a férfiak között vannak melegek:
    \begin{center}
    
        $\exists f(F(f) \land \exists x(F(x) \land V(f,x)) \land \nexists y(N(y) \land V(f,y))) \land$
        $\nexists n(N(n) \land \exists z(N(z) \land V(n,z)) \land \nexists w(F(w) \land V(n,w)))$
    \end{center}
\end{enumerate}

\end{document}